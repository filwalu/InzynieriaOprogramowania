\chapter*{Wstęp}

Sieci oparte o stos TCP/IP są dominującym rozwiązaniem w Internecie i wykorzystywane w sieciach lokalnych i rozległych. Niestety pula adresów IPv4 wyczerpała się \cite{www-1} powodując konieczność wdrażania IPv6. Rysunek \ref{fig:plotend} przedstawia przewidywania dalszego wykorzystania adresacji IPv4.

\begin{figure}[!ht]
	\centering
		\includegraphics[width=16cm]{figures/plotend.png}
	\caption{\footnotesize Przewidywania dotyczące alokacji adresów IPv4. Żródło: \cite{www-1}}
	\label{fig:plotend}
\end{figure}

Sukcesywne wdrażanie IPv6 stało się faktem. Są kraje w których ponad połowa użytkowników Internetu używa IPv6 \cite{www-2}. Tabela \ref{tab:KrajeZ} przedstawia dane krajów w których wdrożenie jest najbardziej zaawansowane oraz polski.

\begin{table}[!ht]
	\caption{\footnotesize Kraje z najbardziej zaawansowanym stanem wdrozenia IPv6.}
		\begin{center}
			\begin{small}
				\begin{tabular}{|l|l|l|l|l|l|l|}
					\hline
					Index&ISO-3166 &Internet Users&V6 Use &V6 Users &Population&Country \\
							 & Code &                 & ratio & (Est) &   & \\
					\hline
					1&BE&10228360&58.14&5946437&11557470&Belgium \\
					2&IN&475642226&52.49&249647006&1366788008&India \\
					3&US&290930388&42.11&122497974&328734902&United States of America \\ 
					4&DE&72552125&41.35&30002918&82445597&Germany \\
					\hline
				
				\end{tabular}
			\end{small}
			\end{center}
	\label{tab:KrajeZ}
\end{table}